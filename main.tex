\documentclass{article}
\usepackage{graphicx}
\graphicspath{{recursos/}}

% cor das caixas
\usepackage{xcolor}
\definecolor{vermelho}{RGB}{224,25,26}

% fontes de texto e ambiente matemático
\usepackage{avant}
\usepackage{mathptmx}

% pacotes de configuração
\usepackage{config/estrutura/headers}
\usepackage[corlink=vermelho]{config/estrutura/fontes}
\usepackage[cor=vermelho]{config/estrutura/ambientes}
\usepackage{config/estrutura/margens}
\usepackage[cor=vermelho]{config/estrutura/secoes}
\usepackage[logo=logo.png]{config/estrutura/titulo}

% pacotes extras
\usepackage{caption, subcaption, pdfpages, wrapfig}
\usepackage{circuitikz, graphics, float}

% começa a seção no `0`
\setcounter{section}{-1}



\hypersetup{
    pdftitle  = {Gráficos no Origin},
    pdfauthor = {Tiago de Paula}
}

\title{Criação e Formatação de Gráficos com o}\softwarelogo
\author{\href{mailto:t187679@dac.unicamp.br}{Tiago de Paula}}
\date{}


\begin{document}
    \maketitle

    O \software é um dos \textit{softwares} mais tradicionais e mais robustos na área de análise gráfica de dados. Inicialmente foi densenvolvido para ser usado com microcalorímetros, mas seu foco foi mudando para áreas de pesquisa em geral e suas funcionalidades acompanharam, se tornando cada vez mais genéricas. Por se tratar de um produto pago, a sua qualidade é bem confiável e ele vem recheado com as mais variadas ferramentas para desenhar gráficos e analisar os dados envolvidos.

    Nesse tutorial, no entanto, serão abordadas apenas as ferramentas necessárias para o curso de Física Experimental 3 (\texttt{F 329}). O tutorial parte das funcionalidades mais básicas, nas duas primeiras seções (\nameref{sec:basico} e \nameref{sec:reta}), que serão usadas em todos os gráficos, até algumas partes um pouco mais específicas, de \nameref{sec:escala}, \nameref{sec:caract}, \nameref{sec:multiv} e \nameref{sec:contorno}. As outras seções, \nameref{sec:regres} e \nameref{sec:incert}, que são intermediárias, também serão usadas em vários gráficos de vários experimentos distintos.

    \section{Configurações Básicas} \label{sec:basico}
        \input{textos/basico}

    \section{Apresentação dos Dados} \label{sec:reta}
        \input{textos/reta}

    \section{Regressão Linear} \label{sec:regres}
        \input{textos/regres}

    \section{Barras de Incerteza} \label{sec:incert}
        \input{textos/incert}

    \section{Escala Logarítmica} \label{sec:escala}
        \input{textos/escala}

    \section{Curva Característica} \label{sec:caract}
        \input{textos/caract}

    \section{Gráficos de Múltiplas Variáveis} \label{sec:multiv}
        \input{textos/multiv}

    \section{Curvas de Nível} \label{sec:contorno}
        \input{textos/contorno}

\end{document}
